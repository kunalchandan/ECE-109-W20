\documentclass{article}

\usepackage{amsmath}
\usepackage[margin=0.5in]{geometry}
%% Begin Doc
\begin{document}
\Huge
Lecture 12

\normalsize

\begin{equation}
\begin{split}
\rho_v = \frac{n_v}{v}  &= \frac{n_{total} * 2.29*10^{-5}}{v} \\
                        &= \frac{(n_{Al} * n_v) * 2.29*10^{-5}}{v} \\
&\approx \frac{n_{Al} * 2.29*10^{-5}}{v}
\end{split}
\end{equation}

\begin{multline}
n_{Al} = \frac{m}{A} * N_A \\
A = 26.98 \frac{g}{mol} \\
N_k = 6.022 * 10^{23} \frac{1}{mol} \\
\rho^{25} = 2.90 \frac{g}{cm^3} \\
\rho^{400} = ?
\end{multline}

Insert a picture of a 1x1x1 cm cube here that weighs 2.7g 

\begin{equation}
\begin{split}
m = \rho * v \\
\rho^{600}  &= \frac{2.70g}{(1.008625 cm)^3} \\
            &= 2.63 \frac{g}{cm^3}
\end{split}
\end{equation}

\begin{equation}
\begin{split}
dl  &= L_o \propto (t_1 - t_0) \\
    &= 0.008625cm
\end{split}
\end{equation}

Thus 
\begin{equation}
\begin{split}
\rho_v = 1.34*10^{18} \frac{\text{vacant sites}}{cm^3}\\
n_{Al} = 5.87 * 10^{22} \text{atoms}
\end{split}
\end{equation}

\LARGE
Physical Defects

One Dimensional Imperfections

\large
Burgers Vector

\normalsize
Similar to a closed loop in physics. This is the Edge dislocation designated by a \"T\" shape
\begin{enumerate}
    \item Create a stepwise loop around a perfect structure
    \item Now do so around the defect. The vector b that is required to end back up at the start is the magnitude of the defect.
\end{enumerate}

\large
Screw Dislocation

\normalsize
Comes as a result of shearing in the structure. Must step down after one turn of the loop.

\LARGE
Physical Defects

Two Dimensional Imperfections

\normalsize
\textbf{Planar defects:} interfaces between crystals

Can form during the following solidification processes.
\begin{enumerate}
    \item solidification start swith molten material
    \item crystals grow until they meet
    \item Grain boundaries form when the crystals meet
\end{enumerate}

\Large
Grain Boundaries

\normalsize
Twin boundary: like a mirror reflection of the crystal.

Tilt boundary: small tilt between crystal interfaces.


\LARGE
Physical Defects

Three Dimensional Imperfections

\normalsize
Orderliness is defined over 2 ranges:
\begin{enumerate}
    \item Short range ordered. Comparable to interatimic distances \textbf{SRO}
    \item Orderliness repeated over long distances is called Long range ordered \textbf{LRO}
\end{enumerate}

Amorphous Metals or Metallic Glasses

Melting then cooling very rapidly (quenching). 
This produces amorphous metals, if you melt and reheat again this will produce a crystal again this is called not thermodynamically stable.

This process decreases the atomic packing factor that is responsible for immediately transfering energy.
The gaps between atoms make the performance better. Like diamond that has very low atomic packing factor.


\Huge
Module 5 - Diffusion

\normalsize
Issues to address:
\begin{enumerate}
    \item How does diffusion occur?
    \item What does diffusion depend on?
    \item How can diffusion rate be predicted?
    \item Why is this important?
\end{enumerate}

\textbf{Definition:} Diffusion is the movement of atoms from an area of high concentration to a lower concentration area. This is a time dependant process.

\Large
Diffusion Mechanisms

\normalsize
Gases \& Liquids: Diffusion is easy due to the large spaces betweens atoms.

Solids:
\begin{enumerate}
    \item Diffusion is difficult
    \item Nearly impossible in crystal structures
    \item Point defects are required for diffusion to occur
\end{enumerate}

\Large
Mechanism \#1: Vacancy Diffusion

\normalsize
\begin{enumerate}
    \item Atoms exchange with vacancies
    \item Applies to substututional impurities
    \item Rate depends on: 
    \item Number of vacancies
    \item Activation energy to exchange
\end{enumerate}


\Large
Mechanism \#2: Interstitial Diffusion

\normalsize
Smaller atoms can diffuse though the interstitial sites inbetween larger atoms.

\Large
Thermal Production of Point Defects

\normalsize
\begin{enumerate}
    \item The generation of point defects is a thermally activated process
    \item How can we quantitatively determint point defect generation?
\end{enumerate}

The rate of the thermally activated process is given by the \textbf{Arrhenius Equation}

\begin{equation}
\begin{split}
\text{rate} &= C e^{\frac{-Q}{RT}} 5.1 \\
ln(\text{rate}) &= ln(C) - {\frac{-Q}{RT}}
\end{split}
\end{equation}

The line yields intercept and the slope (The pre-exponential constant and activation energy )


\Large
Thermal Activation at the Atomic Scale

\normalsize

\begin{equation}
\begin{split}
\text{rate} &= C e^{\frac{-q}{kT}} 5.1
\end{split}
\end{equation}

Instead using the new Boltzmann Constant.

\Large
Activation Energy

\normalsize
The minimum amount of energy that is required to activate atoms, ions or molecules to make a chemical or physical process occur.


\end{document}
