\documentclass{article}

\usepackage{amsmath}
\usepackage{siunitx}
\usepackage[version=4]{mhchem}
\usepackage[margin=0.5in]{geometry}
%% Begin Doc
\begin{document}
\Huge
Lecture 14

\normalsize

\section{Diffusion into semi-finite solids}

Diffusion into a solid that has a finite face and an infinite body.

THe solution for this is:
\begin{equation}
    \begin{split}
        \frac{c_x -c_0}{c_s - c_0} = 1 - \textit{erf}(\frac{x}{2\sqrt{Dt}})
    \end{split}
\end{equation}
Where:
\begin{enumerate}
    \item $t$ is the diffusion time
    \item $x$ is the position
    \item $c_x$ the concentration of the diffusing species
    \item $c_s$ surface concentration of diffusing species
    \item $c_0$ the initial bulk concentration
\end{enumerate}

Example 5.4:

\begin{equation}
    \begin{split}
    c_x &= 0.6\% \\
    c_0 &= 0.2\% \\
    c_s &= 1.0\% \\
    X &= 1\text{mm} \\
    \frac{c_x-c_0}{c_s-c_0} &= 0.5 \\
    D &= 2.98 \times 10^{-11} \frac{m^2}{s}
    \end{split}
\end{equation}

\begin{equation}
    \begin{split}
        \frac{c_x -c_0}{c_s - c_0} &= 1 - \textit{erf}(\frac{x}{2\sqrt{Dt}}) \\
        \textit{erf}(\frac{x}{2\sqrt{Dt}}) &= 1 - \frac{c_x -c_0}{c_s - c_0} \\ 
        &= 0.5 \\
        t &= 3.68\times 10^4s
    \end{split}
\end{equation}

Example 5.5:

Recalculate above with master plot:

\begin{equation}
    \begin{split}
        \frac{c_x -c_0}{c_s - c_0} &= 0.5 \\
        \Rightarrow \frac{x}{\sqrt{Dt}} &= 0.95 \\
        \Rightarrow t &= 3.72\times 10^4s
    \end{split}
\end{equation}


\section{Diffusivity Equations}

\begin{equation}
    \begin{split}
        D = D_0 E^{\frac{-Q}{RT}}
    \end{split}
\end{equation}
Where:
\begin{enumerate}
    \item $D_0$ preexponential constant
    \item $Q$ the activation energy per mole of diffusing species or for the motion of one diffusing species (J/mol, J, or eV)
    \item $R$ the universal gas constant
    \item $T$ the absolute temperature
\end{enumerate}
In general diffusion occurs faster in BCC than in FCC structure.

We can also have diffusivity data for non-metals. Note that for non-metals the diffusivity rate is many orders of magnitude lower than that of metals.

Example 5.6:

\begin{equation}
    \begin{split}
        D &= D_0 E^{\frac{-Q}{RT}} \\
        D_0 &= 20 \times 10^{-6} \\
        Q &= 1.42 \times 10^5 \\
        D &= ?
    \end{split}
\end{equation}

\begin{equation}
    \begin{split}
        \frac{c_x -c_0}{c_s - c_0} &= 1 - \textit{erf}(\frac{x}{2\sqrt{Dt}}) \\
        \frac{0.35-0.20}{1-0.2} &= 1 - \textit{erf}(\frac{4 \text{mm}}{2\sqrt{D(49.5\text{hours})}}) \\
        \Rightarrow D &= 2.6 \times 10^{-11} \frac{m^2}{s} \\
        T &= \frac{Q}{k ( \ln D_0 - \ln D)} \\
        &= \SI{1260.2}{\kelvin}
    \end{split}
\end{equation}

\section{Diffusion in Semiconducting Materials}
We need to be able to dope our semiconductos to yeild special electronic properties.

\begin{enumerate}
    \item Pre-Deposition
    \begin{itemize}
        \item Impurity atoms (e.g. P or B) are diffusted into silicon often from a gas phase. \SI{900}{\celsius} to \SI{1000}{\celsius}
        \item The concentration of the dopant in gas phase and at silicon surface are kept constant
        \item Semi-infinite diffusion model can be applied.
    \end{itemize}
    \item Drive-in Diffusion
    \begin{itemize}
        \item This treatment is used to transport atoms farther into stilicon to provide more suitable concentration distribution without increasing impurity content
        \item Temperature is increased upto \SI{1200}{\celsius}
        \item Treatment applies a \ce{SiO2} layer to prevent escape of impurity.
    \end{itemize}
\end{enumerate}

\section{Drive-in Diffusion}
\begin{enumerate}
    \item The impurity atoms introduced are confined to a very thin layer of the silicon
    \item Then the solution to Flick's second law is:
    \begin{equation}
        \begin{split}
            C(x,t) = \frac{Q_0}{\sqrt{\pi D_d t}}e^{\frac{-x^2}{4D_d t}}
        \end{split}
    \end{equation}
    Where:
    \begin{itemize}
        \item $C(x,t)$ is the impurity concentration at position and time.
        \item $D_d$ is the diffusion coefficient in the drive-in step
        \item $Q_0$ is the total amount of impurities per unity area in the solid that were introduced during the pre-deposition treatment which is:
        \begin{equation}
            \begin{split}
                Q_0 - 2 C_s \sqrt{\frac{D_p t_p}{\pi}}
            \end{split}
        \end{equation}
        Where
        \begin{itemize}
            \item $C_s$ is the surface concentration for the pre-deposition step
            \item $D_p$ is the diffusion coefficient in the pre-deposition step
            \item $t_p$ is the pre-deposition treatment time
        \end{itemize}
    \end{itemize}
\end{enumerate}


Junction Depth, $x_j$: The Depth at which tht ediffusing impurity concentration is just equal to the background concentration of that impurity in the silicon.

\begin{equation}
    \begin{split}
        x_j = \sqrt{(4D_d t_d)\ln(\frac{Q_0}{C_B \sqrt{\pi D_d t_d}})}
    \end{split}
\end{equation}

Example 5.7:

a)
\begin{equation}
    \begin{split}
        Q_0 &= 2C_s\sqrt{\frac{D_p t_p}{\pi}} = 3.44 \times 10^{18} \frac{\text{atoms}}{m^2} \\
        t_p &= 30\text{mm} \times 60 \frac{s}{m} \\
        C_s &= 3 \times 10^{26} \\
        D_p &= D_0 e^{\frac{-q}{kt}} \\
        &= 5.73 \times 10^{-20} \frac{m^2}{s}
    \end{split}
\end{equation}

b)
\begin{equation}
    \begin{split}
        x_j &= \sqrt{(4 D_d t_d) \ln (\frac{Q_0}{C_B \sqrt{\pi D_d t_d}})} \\ 
        &= 2.19 \times 10^{-6} m \\
        &= 2.19 \mu m
    \end{split}
\end{equation}

\begin{equation}
    \begin{split}
        D_d = D_0 e^{\frac{q}{kT_d}} = 1.51 \times 10^{-17}\frac{m^2}{s}
    \end{split}
\end{equation}

c)
\begin{equation}
    \begin{split}
        C(x, t) = \frac{Q_0}{\sqrt{\pi D_d t}} e^{\frac{-x^2}{4D_d t}} = 5.90 \times 10^23 \frac{\text{atoms}}{m^3}
    \end{split}
\end{equation}

\end{document}
