\documentclass{article}

\usepackage{amsmath}
\usepackage{siunitx}
\usepackage[version=4]{mhchem}
\usepackage[margin=0.5in]{geometry}
%% Begin Doc
\begin{document}
\Huge
Lecture 15

\normalsize

Diffusivity is dependant on the structure of the region.
In general $D_\text{volume} < D_\text{gran boundary} < D_\text{surface}$

Example 5.8
\begin{equation}
    \begin{split}
        \frac{C_x -C_0}{C_s-C_0} = 1 - erf(\frac{x}{2\sqrt{Dt}})
    \end{split}
\end{equation}
a)
\begin{equation}
    \begin{split}
        C_x &= 0.01 C_s \\
        C_0 &= 0 \\
        \frac{0.01 C_s}{C_s} &= 0.01 \\
        erf(\frac{x}{2\sqrt{D_0t}}) &= 0.99 \\
        \frac{x}{2D_0t} &= 1.824 \\
        x &= 2.19 \times 10^{-3} m \\
        &= 2.19 mm
    \end{split}
\end{equation}
b)
\begin{equation}
    \begin{split}
        erf(\frac{x}{2\sqrt{D_v t}}) &= 0.99 \\
        x &= 2.19 \times 10^{-5} m \\
        &= 21.9 \mu m
    \end{split}
\end{equation}

\section{Module 6: Band theory and Electrical Properties}

\subsection{Electrical Conduction}
Ohm's Law: $ I = \frac{V}{R} $

Resistivity: Inherent property of material: $\rho = \frac{RA}{l}$

Conductivity: Inverse of Resistivity: $\sigma = \frac{1}{\rho}$

\subsection{Charge Carrier Mobility, \mu}
\begin{itemize}
    \item Charge Carriers
    \begin{itemize}
        \item An electron in a metal
        \item An electron or electron hole in a semiconductor
        \item An ion in an ionic solid or salt solution
    \end{itemize}
    \item Charge Carrier mobility
    \begin{itemize}
        \item The Carrier mobility characterizes how quickly (drift velocity $v_d$) a carrier can move though a metal or semiconductor when pulled by an E field.
        \item $\mu = \frac{v_d}{E}$
    \end{itemize}
    \item Relationship between conductivity and mobility
    \begin{itemize}
        \item $\sigma = nq\mu$
        \item $n$ the density of charge carriers
        \item $q$ the charge carried by each carrier
        \item $\mu$ the mobility of each carrier
    \end{itemize}
    \item When both positive and negative charge carriers contribute to conduction
    \begin{itemize}
        \item $\sigma = n_e q_e\mu_e + n_p q_p\mu_p$
    \end{itemize}
\end{itemize}

Example 6.1
\begin{equation}
    \begin{split}
        \sigma = nq\mu \\
        n = \frac{\sigma}{q\mu} = \frac{58 \times 10^{6}\Omega^{-1} m^{-1}}{0.16\times 10^{-18}C \times 3.5 \times 10^{-3}\frac{m^2}{vs}} = 1.04 \times 10^{29} m^{-3} \\
        n_{Cu} = \frac{\rho}{A_{An}}\times N_A = \frac{8.969/\text{cm}^3 \times 10^6 \frac{cm^3}{nm^3}}{63.5 \frac{g}{mol}}\times 6.022 \times 10^{23} \frac{\text{atoms}}{\text{mol}} = 0.84 \times 10^{29} Cu \frac{\text{atoms}}{\text{m}^3} \\
        \frac{n}{n_{Cu}} = \frac{1.04}{0.84} = 1.24
    \end{split}
\end{equation}

\subsection{Electron States and Energy bands in solids}
In Module 2 we learned about electron structure in single atoms, the same Pauli Exclusion Principle must be applied to multiple atoms.



\begin{equation}
    \begin{split}
    \end{split}
\end{equation}
\end{document}
